% -*- mode: noweb; noweb-default-code-mode: R-mode; -*-
%\VignetteIndexEntry{Rgnuplot Manual}
\documentclass[a4paper,10pt]{amsart}
\usepackage{ucs}
\usepackage[utf8x]{inputenc}
\usepackage[T1]{fontenc}
\usepackage{graphicx}
\usepackage{sidecap}
\setlength{\captionindent}{0pt}
\usepackage{url}
\usepackage[parfill]{parskip}

\renewcommand{\floatpagefraction}{0.8}

\title{Rgnuplot Manual}
\author{Jose Gama}

\date{$ $Id: RgnuplotManual.Rnw 114 2014-04-17 14:22:36Z JoseGama $ $
  processed with Rgnuplot
1.0.1
in R version 3.0.3 (2014-03-06) on \today}
\usepackage{Sweave}
\begin{document}
\input{RgnuplotManual-concordance}

\setkeys{Gin}{width=0.55\linewidth}


\maketitle

\tableofcontents

\section{Installing gnuplot}
\subsection{Installing gnuplot on Linux\\}

On Debian Linux derivatives:\\
\begin{verbatim}sudo apt-get install gnuplot\end{verbatim}

On RedHat Linux derivatives:\\
\begin{verbatim}yum install gnuplot*\end{verbatim}

How to install GNUPLOT+TikZ on Ubuntu:\\
Check GNUPLOT+TikZ For Stunning LaTeX Plots by Tim Teatro\\
\url{http://www.timteatro.net/2010/07/18/gnuplottikz-for-stunning-latex-plots/}

Building from CVS and more:\\
\url{http://www.gnuplot.info/development/}

\subsection{Installing gnuplot on Windows\\}

Download gnuplot for Windows:\\
\url{http://www.gnuplot.info/download.html}

Choose the MinGW Windows binaries built by Tatsuro Matsuoka. Install it.

On Windows 7:

Control Panel/System and Security/System\\
System Properties/Environment Variables/Edit System Variable\\
Add to variable "Path" the path to gnuplot, by default:\\
\begin{verbatim}C:\Program Files\gnuplot\bin\end{verbatim}

Building from CVS and more:\\
\url{http://www.gnuplot.info/development/}


\subsection{Installing gnuplot on MacOSX\\}

Download the gnuplot source:\\
\url{http://sourceforge.net/projects/gnuplot/files/gnuplot/}

Unzip to a folder created for gnuplot.

Edit Makefile (sub folder shlib in the ReadLine source folder), search for -dynamic and change it to -dynamiclib

Type in a console:
\begin{verbatim}make install\end{verbatim}

Change the directory to the gnuplot source folder, type:
\begin{verbatim}./configure --with-readline=< path of the gnuplot folder>
make install\end{verbatim}

Source of this info:
How to install gnuplot in Mac OS X lion
\url{http://bhou.wordpress.com/2011/09/13/how-to-install-gnuplot-in-mac-os-x-lion/}

Building from CVS and more:
\url{http://www.gnuplot.info/development/}

\end{document}
